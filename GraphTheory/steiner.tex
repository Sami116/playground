% !TEX program = xelatex
\documentclass[cyan,normal,cn]{elegantnote}
\usepackage{amsmath}
\usepackage{array}
\usepackage{courier}
\usepackage{xcolor}
\definecolor{light-gray}{gray}{0.95}
\newcommand{\code}[1]{\colorbox{light-gray}{\texttt{#1}}}
\newfontfamily\courier{Courier New}
\lstset{linewidth=1.1\textwidth,
	numbers=left,
	basicstyle=\small\courier,
	numberstyle=\tiny\courier,
	keywordstyle=\color{blue}\courier,
	commentstyle=\it\color[cmyk]{1,0,1,0}\courier, 
	stringstyle=\it\color[RGB]{128,0,0}\courier,
	frame=single,
	backgroundcolor=\color[RGB]{245,245,244},
	breaklines,
	extendedchars=false, 
	xleftmargin=2em,xrightmargin=2em, aboveskip=1em,
	tabsize=4, 
	showspaces=false
	basicstyle=\small\courier
}
\title{Steiner 树综述}
\date{\today}

\begin{document}
\author{
    2021140616 于海鑫 \\
	2021111302 班
}
\maketitle

斯坦纳树是组合优化领域中最为知名的问题之一,其历史非常悠久。最早可以追溯到费马在 17 世纪提出的费马问题。直至今日,无数数学家对于这一问题提出了优化的算法,也有很多专著专门描述这一问题。

\section{历史}

简单来说,斯坦纳树问题的目标是求解出使得每个节点的路径长度之和最小的树。这一目标与最小生成树问题非常类似,不同的是斯坦纳树允许添加新的节点,而最小生成树不允许添加新的节点。这一问题的特殊情况最早由费马于 1640 年之前提出,也就是费马问题:

给定一个平面上的三个点,在平面内找出一个点,使得这个点到给定的三个点的距离之和最小。

在当时的 17 世纪初期,由于解析几何刚刚开始出现,求极值问题也刚刚萌芽,费马问题也成了难题,一时间竟然没有人能给出合理的求解方式。过了十几年后,这一问题被梅森引入到意大利,最后终于被数学家托里彻利解决。费马问题很容易被推广,瑞士数学家斯坦纳将其推广为:

\textit{给定任意 $N$ 个在同一个平面上的点 $A_1, A_2 \cdots A_N$,求得一点 $P$,使得 $P$ 到 $A_1, A_2 \cdots A_N$ 的距离之和最小。}

在同期,还有英国数学家辛普森将其推广为一个加权问题,也就是:

\textit{对于给定的三个正数 $a, b, c$ 以及三个点 $A_1, A_2, A_3$,求得一点 $P$,使得 $a|PA_1| + b|PA_2| + c|PA_3|$ 最小。}

最终,这一问题被亚尔尼克等人推广为现在广为熟知的斯坦纳树问题:

\textit{对于平面上给定的 $N$ 个点 $A_1, A_2, \cdots, A_N$,求出一个链接这 N 个点的最小网络。}

对斯坦纳树问题的形式化定义如下:

令 $G$ 为无向有权图 $G=(V,E)$,$V$ 为其点集,$E$ 为其边集。边集 $V$ 是由 \textit{终端节点} $T \subseteq V$ 以及潜在的 \textit{斯坦纳节点} $S=V \setminus T$ 组成的不相交集。权重函数 $c \colon E \rightarrow R^{+}$ 为每个边分配了一个非负权重。

我们的目标是找到包含全部终端节点的成本最低子图 $G_T$ 也就是 $S \subseteq V_T \subseteq V, E_T \subseteq E$ 同时 $min(\sum_{e \in E_T} c(e))$。斯坦纳问题是 $\mathcal{NP}$-hard 的组合优化问题。

\section{算法}

\subsection{暴力算法}
对于斯坦纳树问题来说,最简单的算法就是进行枚举,枚举出所有的可能的结果,然后选择其中的最小值。历史上对于这一问题的简单情形都给出了不错的枚举方案。

对于三个点的斯坦纳树问题,也就是费马问题,如同上文所说,意大利的托丽彻利给出了解决方案:在三角形的三条边上作等边三角形,并对每一个三角形作外接圆,这三个圆的交点即为斯坦纳节点。

四个点的斯坦纳树问题比三个点的情况复杂的多,对于三个点的斯坦纳树,仅仅存在 4 种可能的连接方式。但是四个点的斯坦纳树光连接方式就存在 31 种,几乎无法给出一个简洁明了的求解方式。最终 1978 年波拉克给出了一个定理,列出了求解方式。但是随着点的个数的增大,需要求解的可能性大大增加,六个点的斯坦纳树问题的可能的解有 5625 种,而八个点的斯坦纳树问题的可能解有 2643795 种。可以看出来,随着数字的迅速增大,枚举已经几乎是不可能的了,这也是 $\mathcal{NP}$-hard 问题的主要特点。目前对于这类问题,我们有两种方法来简化:

\begin{itemize}
	\item 对点的性质加以限制
	\item 放弃寻求最优解,转而寻求近似最优解
\end{itemize}

因为实际应用的时候点的性质很难保证,同时很多时候又不是要求绝对的最优解,只是要一个成本尽可能小的解。因此,寻求近似最优解的方法被研究的更多一些,下面简要介绍一下近似最优解的方法。

\subsection{启发式算法}

启发式算法的思想就是放弃求解最优解,而是寻求近似最优解。最简单的启发式算法就是把最小生成树视为斯坦纳树,求解最小生成树的算法的时间复杂度可以达到 $O(N \log N)$,与排序相当,可以接受。当然后期也有人对于这一算法提出改进,例如选择一些中间点,然后把它们作为斯坦纳节点,这样可以进一步使得结果接近于真正的斯坦纳树。除去基于最小生成树的启发式算法,还有一种采用模拟退火进行求解的启发式算法,这一算法被广泛地用于 $\mathcal{NP}$-hard 问题的求解,其借助了随机数的力量,配合局部最优解,一步一步尽可能的逼近真正的最优解。与之类似的算法还有集群算法等等。

除去这些专为斯坦纳树问题设计的算法外,也有一些将这一问题转化为其他问题,并套用对应的解决方法的算法出现,最常见的是将这一问题转换为另一个知名的优化问题“旅行商问题”,之后套用对应的算法求解。


\section{应用}

斯坦纳树问题有着诸多实际应用,以下列出一些与计算机相关的应用。

\subsection{集成电路后端设计}

在集成电路的后端设计过程中,当通过布局确定各个组件的位置之后,需要利用组件之间的空闲部分进行布线,将组件连接起来,使用了相同的信号作为输入/输出的端口需要连接到同一个信号网上。为了提高性能,同时节约成本,信号网络应当尽量短。显然这个问题可以被简化为斯坦纳树问题。

\subsection{计算机网络路由问题}

一个典型的多播网络可以被抽象为一个图 $G = (V, E, c)$ ,其中每个节点 $v$ 可以接收到一个或多个信号,而每个信号都可以被发送到一个或多个节点,也就是说 $V$ 可以被分为两部分,源节点 $s$,以及目的节点 $S \in V \setminus \{s\}$。对于多播网络,确定其权重会比较复杂,权重通常会和网络的延迟、价格有关。对于这种网络,存在两种很常见的情况:$|S| = 1$,也就是单播;$|S| = |V| - 1$,也就是广播。路由树 $T$ 是 $G$ 的一个子图,以 $s$ 为源节点,同时对于 $S$ 中的节点都是连通的。通常我们要求求出最优路由。对于最优路由,我们要求出一个最小的路由树 $T$,使得 $min(\sum_{e \in E_T} c(e))$ 最小。求出最优路由树 $T$ 可以利用斯坦纳树问题。

\end{document}